\geometry{a4paper, top=25mm, left=20mm, right=20mm, bottom=25mm}

% Warning box
\newtcolorbox{warning}[1][]
{
  colframe=yellow!50!red,
  colback=yellow!30,
  coltitle=red!20!black,
  title={Achtung!},
  #1
}

\newtcolorbox{instruction}[1][]
{
  colframe=cyan!50!blue,
  colback=cyan!40,
  coltitle=blue!20!black,
  title={Jetzt Du!},
  #1
}

\newcommand{\hrefnote}[2]{\href{#1}{#2}\footnote{\url{#1}}}
\newcommand{\hrefnoteinline}[2]{\href{#1}{#2} (\url{#1})}

%% Define an Arduino style fore use later %%
\lstdefinestyle{myArduino}{
  language=Arduino,
  %% Add other words needing highlighting below %%
  morekeywords=[1]{},                  % [1] -> dark green
  morekeywords=[2]{FILE_WRITE},        % [2] -> light blue
  morekeywords=[3]{SD, File},          % [3] -> bold orange
  morekeywords=[4]{open, exists},      % [4] -> orange
  %% The lines below add a nifty box around the code %%
  frame=shadowbox,
  rulesepcolor=\color{arduinoBlue},
}

\lstMakeShortInline[style=myArduino]|
\lstset{style=myArduino}

% Patch from https://tex.stackexchange.com/questions/451532/recent-issues-with-lstlinebgrd-package-with-listings-after-the-latters-updates
\makeatletter
\let\old@lstKV@SwitchCases\lstKV@SwitchCases
\def\lstKV@SwitchCases#1#2#3{}
\makeatother
\usepackage{lstlinebgrd}
\makeatletter
\let\lstKV@SwitchCases\old@lstKV@SwitchCases

\lst@Key{numbers}{none}{%
    \def\lst@PlaceNumber{\lst@linebgrd}%
    \lstKV@SwitchCases{#1}%
    {none:\\%
     left:\def\lst@PlaceNumber{\llap{\normalfont
                \lst@numberstyle{\thelstnumber}\kern\lst@numbersep}\lst@linebgrd}\\%
     right:\def\lst@PlaceNumber{\rlap{\normalfont
                \kern\linewidth \kern\lst@numbersep
                \lst@numberstyle{\thelstnumber}}\lst@linebgrd}%
    }{\PackageError{Listings}{Numbers #1 unknown}\@ehc}}
\makeatother

% https://tex.stackexchange.com/questions/8851/how-can-i-highlight-some-lines-from-source-code


\newcommand{\sidebyside}[1]{
\begin{minipage}{.5\textwidth}
  \begin{center}
    \includegraphics[width=.8\textwidth,
        % cut 1 cm from the bottom
        trim=0 .5cm 0 0, clip
    ]{schaltung/#1/#1_schem.png}
  \end{center}
\end{minipage}%
\begin{minipage}{.5\textwidth}
  \begin{center}
    \includegraphics[width=.8\textwidth,
        % cut 1 cm from the bottom
        trim=0 0.5cm 0 0, clip
    ]{schaltung/#1/#1_bb.png}
  \end{center}
\end{minipage}
}